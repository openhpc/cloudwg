%%
%% This is file `sample-sigconf.tex',
%% generated with the docstrip utility.
%%
%% The original source files were:
%%
%% samples.dtx  (with options: `sigconf')
%% 
%% IMPORTANT NOTICE:
%% 
%% For the copyright see the source file.
%% 
%% Any modified versions of this file must be renamed
%% with new filenames distinct from sample-sigconf.tex.
%% 
%% For distribution of the original source see the terms
%% for copying and modification in the file samples.dtx.
%% 
%% This generated file may be distributed as long as the
%% original source files, as listed above, are part of the
%% same distribution. (The sources need not necessarily be
%% in the same archive or directory.)
%%
%% The first command in your LaTeX source must be the \documentclass command.

% use manuscript for submission (1-column format)
%\documentclass[manuscript,screen]{acmart}
% however, short papers should be 3-4 pages in 2-column format
\documentclass[sigconf,screen]{acmart}

\usepackage{listings}
\lstset{basicstyle=\footnotesize\ttfamily,language=Python, tabsize=1, frame=single}

\usepackage{caption}
\usepackage{subcaption}
\usepackage{amsmath}
\usepackage{algorithm}
\usepackage[noend]{algpseudocode}
\usepackage{booktabs}
\usepackage[flushleft]{threeparttable}
\usepackage{lipsum}

\makeatletter
\def\BState{\State\hskip-\ALG@thistlm}
\makeatother
%%%% As of March 2017, [siggraph] is no longer used. Please use sigconf (above) for SIGGRAPH conferences.

%%%% Proceedings format for SIGPLAN conferences 
% \documentclass[sigplan, anonymous, review]{acmart}

%%%% Proceedings format for SIGCHI conferences
% \documentclass[sigchi, review]{acmart}

%%%% To use the SIGCHI extended abstract template, please visit
% https://www.overleaf.com/read/zzzfqvkmrfzn

%%
%% \BibTeX command to typeset BibTeX logo in the docs
\AtBeginDocument{%
  \providecommand\BibTeX{{%
    \normalfont B\kern-0.5em{\scshape i\kern-0.25em b}\kern-0.8em\TeX}}}

%% Rights management information.  This information is sent to you
%% when you complete the rights form.  These commands have SAMPLE
%% values in them; it is your responsibility as an author to replace
%% the commands and values with those provided to you when you
%% complete the rights form.

\copyrightyear{2021} 
\acmYear{2021} 
\setcopyright{acmlicensed}
\acmConference[PEARC '21]{Practice and Experience in Advanced Research Computing}{July 19--22, 2021}{Virtual}
\acmBooktitle{Practice and Experience in Advanced Research Computing (PEARC '21), July 19--22, 2021}
\acmPrice{15.00}
% have to udpate this doi if accepted
\acmDOI{10.1145/1122445.1122456}
\acmISBN{978-1-4503-6689-2/20/07}


%%
%% Submission ID.
%% Use this when submitting an article to a sponsored event. You'll
%% receive a unique submission ID from the organizers
%% of the event, and this ID should be used as the parameter to this command.
%%\acmSubmissionID{123-A56-BU3}

%%
%% The majority of ACM publications use numbered citations and
%% references.  The command \citestyle{authoryear} switches to the
%% "author year" style.
%%
%% If you are preparing content for an event
%% sponsored by ACM SIGGRAPH, you must use the "author year" style of
%% citations and references.
%% Uncommenting
%% the next command will enable that style.
%%\citestyle{acmauthoryear}



%%
%% end of the preamble, start of the body of the document source.
\begin{document}

%%
%% The "title" command has an optional parameter,
%% allowing the author to define a "short title" to be used in page headers.
%\title{Processing of Passive Digital Phenotyping Location Data}
\title{Deploying OpenHPC on AWS Cloud Resources}

%%
%% The "author" command and its associated commands are used to define
%% the authors and their affiliations.
%% Of note is the shared affiliation of the first two authors, and the
%% "authornote" and "authornotemark" commands
%% used to denote shared contribution to the research.
\author{Chris Simmons}
\email{christopher.simmons@utdallas.edu}
\affiliation{\institution{University of Texas at Dallas}}
\author{et. al}
\email{karl@oden.utexas.edu}
\affiliation{\institution{Oden Institute for Computational Engineering and
    Sciences and more foo}}


%%
%% By default, the full list of authors will be used in the page
%% headers. Often, this list is too long, and will overlap
%% other information printed in the page headers. This command allows
%% the author to define a more concise list
%% of authors' names for this purpose.
%\renewcommand{\shortauthors}{Zachary and Karl}

%%
%% The abstract is a short summary of the work to be presented in the
%% article.
\begin{abstract}
  With the growing interest in using cloud-based resources in support of
  traditional HPC workloads, there is a need to aggregate development and
  performance libraries from multiple sources for use in dynamic cloud
  environments. This paper highlights the use of OpenHPC, a Linux Foundation
  project combined with provisioning mechanisms available within AWS to provide
  a modular, open-source HPC development enviroment for HPC workloads.  We
  highlight instantation on AWS using aarch64 instances and the openSUSE Leap
  distribution using the SLURM resource manager to provide a familiar HPC
  operating environment running the cloud that can support multi user-tenancy.
  MPI micro-benchmarks are carried out using OpenHPC's latest build
  configurations supporting the AWS elastic fabric adapter.
\end{abstract}



%%
%% The code below is generated by the tool at http://dl.acm.org/ccs.cfm.
%% Please copy and paste the code instead of the example below.
%%
\begin{CCSXML}
\end{CCSXML}

%\ccsdesc[500]{Computer systems organization~Embedded systems}
%\ccsdesc[300]{Computer systems organization~Redundancy}
%\ccsdesc{Computer systems organization~Robotics}
%\ccsdesc[100]{Networks~Network reliability}

%%
%% Keywords. The author(s) should pick words that accurately describe
%% the work being presented. Separate the keywords with commas.
\keywords{OpenHPC, cloud computing and networking, HPC workloads}

%%
%% This command processes the author and affiliation and title
%% information and builds the first part of the formatted document.
\maketitle

\section{Introduction}
Motherhood, meet apple pie.

\subsection{Insallation Overview}
Walk them thru an overview of how to do the installation. What tools are being
used...pointer to working docs.

\subsection{Instance Overview}

Discussion/recommendation on instance types that are likely the most relevant
for HPC use; comment on bursty instances.

\lipsum[4]

\section{Results}
\lipsum[6]
\subsection{Cluster assembly}
\begin{itemize}
\item maybe highlight some meaurements for the time it takes to build the
  cluster at AWS from scratch?
\item highlight time to launch job with dynamic resource manager power on/off
  (ie, how long to launch a job if no nodes are available)
\item resulting final image size to stash if wanting to save?
\end{itemize}

\subsection{MPI microbenchmarks}
\begin {itemize}
\item demonstrate reasonable latencies and effective bandwidth with EFA
\item compare mpich/openmpi
\item contrast with distro provided mpi if possible?
\end{itemize}

\section{Conclusion}
\lipsum[8]

%%
%% The acknowledgments section is defined using the "acks" environment
%% (and NOT an unnumbered section). This ensures the proper
%% identification of the section in the article metadata, and the
%% consistent spelling of the heading.
\begin{acks}
The authors would like to thank members of the OpenHPC cloud working group and
technical steering committee who have provided guidance and input on efforts to
enable usage of OpenHPC-based builds on cloud resources.
\end{acks}

%%
%% The next two lines define the bibliography style to be used, and
%% the bibliography file.
\nocite{*}
\bibliographystyle{ACM-Reference-Format}
\bibliography{refs}

%%
%% If your work has an appendix, this is the place to put it.

\end{document}
\endinput
%%
%% End of file `sample-sigconf.tex'.
